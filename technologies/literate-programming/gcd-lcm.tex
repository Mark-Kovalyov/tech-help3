\input cwebmac

\N{1}{1}LCD-LCM

In mathematics, the greatest common divisor (gcd) of two or
more integers, which are not all zero, is the largest
positive integer that divides each of the integers.
For example, the gcd of 8 and 12 is 4

\Y\B${*{}$\6
\8\#\&{include} \.{<stdio.h>}\6
${}}\&{int}{}$ \\{gcd}(\&{int} \|a${},\39{}$\&{int} \|b)\1\1\2\2\6
${}\{{}$\1\6
\&{if} ${}(\|b\I\T{0}){}$\1\5
\&{return} \\{gcd}${}(\|b,\39\|a\MOD\|b);{}$\2\6
\&{else}\1\5
\&{return} \|a;\2\6
\4${}\}{}$\2\par
\fi

\M{2}
The greatest common divisor can be used to find the least
common multiple of two numbers when the greatest common
divisor is known, using the relation

\Y\B${*}\&{int}{}$ \\{lcm}(\&{int} \|a${},\39{}$\&{int} \|b)\1\1\2\2\6
${}\{{}$\1\6
\&{return} ${}(\|a*\|b)\MOD\\{gcd}(\|a,\39\|b);{}$\6
\4${}\}{}$\2\par
\fi

\N{1}{3}Index.
\fi

\inx
\fin
\con
